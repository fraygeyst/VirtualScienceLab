% !TeX root = ../pythonTutorial.tex
\chapter{Ausblick}

 Ob und wie es mit dem Projekt Virtual Science Lab weitergehen wird, steht zum Zeitpunkt dieser Abgabe noch in den Sternen, jedoch ist es von unserer Seite mehr als w�nschenswert, dass es eine Zukunft f�r das Projekt geben wird. Die aktuellen Versuche sind immer verbesserbar. Allein der Unterschied zwischen dem Bunsenbrenner im Ausgangsprojekt und der neu modellierte zeigen, dass kleine Details einen immensen Unterschied f�r das Gesamterlebnis beitragen k�nnen. Zwar wurde schon sehr auf Details wert gelegt, doch perfekt ist eben doch zuletzt im Ermessen des Betrachters. \newline
 
 Au�erdem ist es sinnvoll, den Android Build erneut anzugehen. Bei diesem besteht das gro�e Problem, verschiedene Ger�te mit unterschiedlichsten Steuerungsarten mit einzubeziehen.  \newline
 
 Au�erdem w�re eine Art Feldtest sinnvoll, in dem getestet wird, ob Sch�ler oder Studenten, die das Virtual Science Lab absolvieren, in einer passenden Pr�fung besser abschneiden k�nnten, als eine Gruppe, die das Labor nicht gesehen hat.  \newline
 
 Der Aufbau der Laborr�ume glich immer mehr einem Prozess, bei dem dieselben Schritte nacheinander abgespult wurden. Zun�chst wird der Raum durch W�nde, Boden und Decke begrenzt, eine T�r wird hinzugef�gt, vor der die Kamera platziert wird, sodass man an der Position der T�r den Raum betritt. Anschlie�end werden falls n�tig Tische angeordnet, so wie die verschiedenen Utensilien die f�r die vorgesehenen Versuche n�tig sind. Zum Schluss werden die Skripte zur Funktionalit�t geschrieben, um so die Versuchsdurchf�hrung zu erm�glichen. Anschlie�end wird der Versuch im Virtual Reality Labor getestet und falls n�tig Bugs notiert, die im Anschluss gefixt werden. Dieser Evaluationsvorgang wird iterativ beliebig oft durchgef�hrt, bis das Ergebnis zufriedenstellend ist.  \newline
 
 