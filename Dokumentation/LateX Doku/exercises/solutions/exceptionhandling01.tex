Bei dem Ablauf des Programms werden alle Ergebnisse in die Textdatei geschrieben.
Lediglich bei dem letzten Durchlauf kommt es zu einer Division mit der Zahl null (11 / 0). Dabei wird die Ausnahme \lstinline$divsion by zero$ ausgel�st und der \lstinline$try-Block$ wird sofort verlassen, um den entsprechenden \lstinline$except-Block$ zu durchlaufen. 
Anschlie�end wird die Textdatei durch die \lstinline$finally$-Anweisung geschlossen und das Programm beendet.
\\ \\
Um das gew�nschte Resultat zu erhalten, verwenden wir zun�chst die \lstinline$range$-Methode um unsere beiden List-Datenstrukturen a und b mit den entsprechenden Werten zu f�llen.
Als Laufvariable wird \lstinline$i$ verwendet und auf den Wert 10 gesetzt, da dies dem letzten Index einer List entspricht.
\lstinputlisting[language=Python, firstline=1,lastline=4]{exercises/src/exceptionhandling01.py}

In dem \lstinline$try-Block$ wird zuerst die Datei mit dem Namen \lstinline$result.txt$ erstellt und ge�ffnet.
Anschlie�end wird in einer while-Schleife die Division durchgef�hrt und das Ergebnis in die Textdatei geschrieben. 
Erst danach wird die Laufvariable dekrementiert und der Vorgang, falls \lstinline$i$ die Bedingung noch erf�llt, ausgef�hrt.
\lstinputlisting[language=Python, firstline=5,lastline=11]{exercises/src/exceptionhandling01.py}

Mit den entsprechenden Fehlerklassen k�nnen die Ausnahmen abgefangen werden.
Eine zus�tzliche \lstinline$finally$-Anweisung sorgt f�r ein ordnungsgem��es Schlie�en unserer ge�ffneten Textdatei.
\lstinputlisting[language=Python, firstline=12,lastline=33]{exercises/src/exceptionhandling01.py}

	
