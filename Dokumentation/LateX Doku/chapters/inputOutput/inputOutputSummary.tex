% !TeX root = ../../pythonTutorial.tex
\section{Zusammenfassung}
\label{filehandling:subsection:zusammenfassungdateienlesenundschreiben}

In diesem Kapitel haben wir uns mit dem lesen und beschreiben einer Datei auseinandergesetzt.
Dies geschieht in Python mithilfe eines \lstinline$fileObject$ um eine Datei zu erstellen, �ndern, l�schen und abzuspeichern.
Dabei kann eine Datei als Textdatei oder Bin�rdatei interpretiert werden. 
Eine der wichtigsten Methoden stellt hierbei die \lstinline$open()$-Methode dar. 
Diese erm�glicht uns das Erstellen, �ffnen, Aktualisieren, Lesen und Beschreiben einer Datei.
Dabei ist zu beachten, dass die Datei direkt nach der Ausf�hrung der gew�nschten Operationen mithilfe der \lstinline$close()$-Methode geschlossen wird.
Um dies nicht zu vergessen, besteht in Python auch die M�glichkeit ein automatisches Schlie�en mit dem \lstinline$With$-Statement zu erwirken. 
Abschlie�end haben wir noch den Zugriff auf die wichtigsten Attribute, die ein \lstinline$fileObject$ besitzt, kennengelernt und uns mit dem Standardisierten Datenaustausch mittels JSON besch�ftigt.