% !TeX root = ../../pythonTutorial.tex
\section{JSON}
\label{filehandling:section:json}

JavaScript Object Notation (JSON) ist ein Format f�r den Austausch von Daten, das unabh�ngig von der Programmiersprache ist. Aufgrund von Konventionen,
die dieses Format mit Programmiersprachen aus der C-Familie, wie C, C++, Java oder Python teilt, liefert es eine Programmierern bekannte Textstruktur.

In Python 3 ist nativ das json-Package enthalten, das das Arbeiten mit
dem JSON-Format erm�glicht. Mithilfe des folgenden Codes binden wir das Package in das Projekt ein. 
\lstinputlisting[language=Python, linerange={1-1,3-5}]{chapters/inputOutput/src/jsonInPython/JsonInPython.py}
\label{filehandling:lst:importpackage}

Ein gegebener JSON-String wird �ber die \lstinline$loads()$-Methode in ein in Python existierendes, entsprechendes Objekt geparst.
In diesem Fall wird ein Dictionary angelegt.
\randnotiz{JSON zu Python}
\lstinputlisting[language=Python, linerange={1-1,7-19}]{chapters/inputOutput/src/jsonInPython/JsonInPython.py}
\label{filehandling:lst:loads}

F�r das Umwandeln eines Python-Objekts in einen JSON-String verwenden wir die 
\lstinline$dumps()$-Methode.
\randnotiz{Python zu JSON}
\lstinputlisting[language=Python, linerange={1-1,21-35}]{chapters/inputOutput/src/jsonInPython/JsonInPython.py}
\label{filehandling:lst:dumps}

Konvertieren wir Python- zu JSON-Objekte, werden diese im\\
JSON-�quivalent (JavaScript) angelegt.

Wenn wir einen Dictionary mit mehreren Schl�ssel-Objekt-Paaren anlegen,
werden wir bei der Ausgabe des JSON-Objekts feststellen, dass diese auf eine Zeile beschr�nkt sind. 
\lstinputlisting[language=Python, linerange={1-1,37-55}]{chapters/inputOutput/src/jsonInPython/JsonInPython.py}
\label{filehandling:lst:format1}

Zur Formatierung unserer Ausgabe verwenden wir die \lstinline$dumps()$-Methode.
Mithilfe des \lstinline$indent$-Parameters k�nnen wir festlegen, ob und wie weit die Textstruktur einger�ckt werden soll. 
Der \lstinline$separators$-Parameter legt die\\
Trennzeichen fest und mit \lstinline$sort_keys=True$ wird die Ausgabe der Schl�ssel lexikografisch sortiert.
\lstinputlisting[language=Python, linerange={1-1,37-37,56-81}]{chapters/inputOutput/src/jsonInPython/JsonInPython.py}  
\label{filehandling:lst:format2}


