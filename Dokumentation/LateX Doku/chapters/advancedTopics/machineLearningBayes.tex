\subsection{Beispiel: Naive Bayes}\label{maschinelleslernen:naivebayes}
Ein weiteres Beispiel aus dem Bereich Klassifikation im maschinellen Lernen ist der Naive Bayes Algorithmus.

Es \randnotiz{Importe}sind wieder einige Importe n�tig.
\lstinputlisting[language=Python,firstline=3,lastline=5]{chapters/advancedTopics/src/machinelearning/bayessample.py}\label{knnsample:lst:bayessample3}

Die \randnotiz{Daten laden}Beispieldaten m�ssen definiert werden.
\lstinputlisting[language=Python,firstline=7,lastline=11]{chapters/advancedTopics/src/machinelearning/bayessample.py}\label{knnsample:lst:bayessample7}

Nun \randnotiz{Modell erzeugen}kann das Modell erzeugt werden um mit den Daten zu trainieren. 
\lstinputlisting[language=Python,firstline=13,lastline=16]{chapters/advancedTopics/src/machinelearning/bayessample.py}\label{knnsample:lst:bayessample13}

Anschlie�end\randnotiz{Vorhersage} k�nnen die Daten vorhergesagt und die Ergebnisse angezeigt werden.
\lstinputlisting[language=Python,firstline=17,lastline=21]{chapters/advancedTopics/src/machinelearning/bayessample.py}\label{knnsample:lst:bayessample17}

Die Ausgabe dieser Vorhersage lautet:
\begin{lstlisting}
[3 4]
\end{lstlisting}


\uebung
\aufgabe{MachineLearning/machinelearning_bayes}
